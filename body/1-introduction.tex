\hspace{24pt}

Add your introduction here.

\section{Mathematic Symbols}
% *** mathematic symbols (use dollar sign to wrap it)***
$C_t$ example of sub
$C^t$ example of power
% more symbols in latex : https://en.wikipedia.org/wiki/Wikipedia:LaTeX_symbols

\section{Equation}
% *** example of equation ***
Below is equation~\ref{eq:example}.
\begin{eqnarray}
        C_t=\{x|x \in P \wedge D(t,x) \leq \delta \wedge A(x) > 0 \}.
        \label{eq:example}
\end{eqnarray}
% more about equation in latex :  https://en.wikibooks.org/wiki/LaTeX/Advanced_Mathematics

\section{Step lists}
\begin{description}
    \item [Step 1.] balabala
    \item [Step 2.] balabala
\end{description}

\section{Algorithm}
% *** example of algorithm ***
% require \usepackage{algorithm}
We proposed algorithm ~\ref{algorithm:example}.

\begin{algorithm}[H]
    \caption{Algorithm Example}
    \label{algorithm:example}
    
    \begin{algorithmic}
        \If {$i\geq maxval$}
            \State $i\gets 0$
        \Else
            \If {$i+k\leq maxval$}
                \State $i\gets i+k$
            \EndIf
        \EndIf
    \end{algorithmic}
\end{algorithm} 
% more about algorithm in latex : https://en.wikibooks.org/wiki/LaTeX/Algorithms (Typesetting using the algorithmicx package)
% http://ftp.yzu.edu.tw/CTAN/macros/latex/contrib/algorithms/algorithms.pdf

\section{Label} \label{subsec: labelName}
This is Section~\ref{subsec: labelName}.

\section{Reference}
Professor Tsai published many good papers~\cite{TsaimhPaper}.

\section{Subsection}
\subsection{one}


\section{Table}
% *** Table ***
Table~\ref{tab:example} is a 3 x 3 table.
\begin{table}[h] 
    \normalsize
    \caption{Table Example} 
    \begin{center} 
        \label{tab:example} 
        \begin{tabular}{ | l | c || r |}
            % use | to draw the vertical line
            % use \hline to draw horizontal line
            \hline
            1 & 2 & 3 \\ \hline\hline
            4 & 5 & 6 \\ \hline
            7 & 8 & 9 \\ 
            \hline
        \end{tabular} 
    \end{center} 
\end{table}
% more about table in latex :  https://en.wikibooks.org/wiki/LaTeX/Tables

\section{Figure}
% *** figure ***
Figure~\ref{fig:example} is flowchart.
\begin{figure}[h]
    \begin{center}
        \includegraphics[width=3.4in]{images/logo.png}
    \end{center}
    \caption{Figure Example}
    \label{fig:example}
\end{figure}

\section{Plot}
% *** plot ***
The results show in Figure~\ref{fig:plot-example}
\begin{figure}[tbp]
    \begin{tikzpicture}
        \begin{axis}[
            width = \textwidth,
            xlabel = {x label},
            ylabel = {y label},
            % scaled y ticks={base 10:2}, 
            legend pos = south east,
            legend style = {font=\tiny},
        ]
            % line 1
            \addplot+[draw = black, mark = *, mark options = {scale=1, fill=white}] table {data/line1.dat};
            % line 2 
            \addplot+[draw = black, mark = square*, mark options = {scale=1, fill=white}] table {data/line2.dat};
            \legend{line1, line2}
        \end{axis}
    \end{tikzpicture}
    \caption{Plot Example}
    \label{fig:plot-example}
\end{figure}
% 