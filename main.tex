% %
% This file is encoded in utf-8
% This file is modified from
% 1. http://exciton.eo.yzu.edu.tw/~lab/latex/latex_note.html
%    (元智大學模版 陳念波老師)
% 2. http://code.google.com/p/ntuthesis/
%    (臺大碩士、博士論文的Latex模板)
%%
\documentclass[12pt,p a4paper]{class/ncku_class}

\usepackage{CJKutf8}  %%% ZZZ %%% macro for Chinese/Japanese/Korean processing
\usepackage{CJKnumb} %%% ZZZ %%% for Chinese numbering capability
\usepackage[nospace]{cite}  % for smart citation
\usepackage{geometry}  % for easy margin settings
\usepackage{class/ncku_style} % 自定義nckuee.sty // cbj
\usepackage{booktabs}

% 插圖套件 graphicx
% 使用者工作流程是用 pdftex 還是 latex + dvipdfmx?
% 視情況而有不同的參數
% 這裡作自動判斷
% 參考自
% http://www.tex.ac.uk/cgi-bin/texfaq2html?label=ifpdf
%%\newcommand\mydvipdfmxflow{dvipdfmx}P
%%\newcommand\mypdftexflow{pdftex}P
%%\ifx\pdfoutput\undefined
%%  % not running pdftex
%%  \usepackage[dvipdfm]{graphicx}
%%  \newcommand\myworkflow{dvipdfmx}  % set the flag for hyperref
%%\else
%%  \ifx\pdfoutput\relax
%%    % not running pdftex
%%    \usepackage[dvipdfm]{graphicx}
%%    \newcommand\myworkflow{dvipdfmx}  % set the flag
%%  \else
%%    % running pdftex, with...
%%    \ifnum\pdfoutput>0
%%      % ... PDF output
%%      \usepackage[pdftex]{graphicx}
%%      \newcommand\myworkflow{pdftex}  % set the flag
%%    \else
%%      %...DVI output
%%      \usepackage[dvipdfm]{graphicx}
%%      \newcommand\myworkflow{dvipdfmx}  % set the flag
%%    \fi
%%  \fi
%%\fi
      \usepackage[pdftex]{graphicx}
      \newcommand\myworkflow{pdftex}  % set the flag

% 增強功能型頁楣 / 頁腳套件
\usepackage{fancyhdr}  % 借用此套件來擺放浮水印
% (佔用了 central header)
% 不需要浮水印的使用者仍可利用此套件,產生所需的 header, footer
%
% 啟動 fancy header/footer 套件
\pagestyle{fancy}
\fancyhead{}  % reset left, central, right header to empty
\fancyfoot[C]{\thepage} %中間 footer 擺放頁碼
\renewcommand{\headrulewidth}{0pt} % header 的直線; 0pt 則無線

% 如果不需要任何浮水印,則請把下列介於 >>> 與 <<< 之間
% 的文字行關掉 (行首加上百分號)
%% 浮水印 >>>
\input{header/ncku_watermark.tex}
%% <<< 浮水印
% 如需額外的頁楣 (header) 或 footer,請在 header/header_footer.tex 裡依例修改
% 它的預設內容是都關掉,可依需要打開
\input{header/header_footer.tex}

%%%%%%%%%%%%%%%%%%%%%%%%%%%%%%
%%%% 非必要的套件,但很實用
\usepackage{amsmath} % 各式 AMS 數學功能
\usepackage{amssymb} % 各式 AMS 數學符號
\usepackage{mathrsfs} %草寫體數學符號,在數學模式裡用 \mathscr{E} 得草寫 E
\usepackage{listings} % 程式列表套件

%%%%%%%%%%%%%%
% // 可自行增加所需套件 // cbj
%\usepackage{subfig}
%%%%%%%%%%%%%%
\usepackage{subfigure}
\usepackage{graphicx} \graphicspath{{images/}}
\usepackage{psfrag} % text replacement in figure
\usepackage{amsmath}
\usepackage{bm}
\usepackage{mathtools}
\usepackage{hyperref}
\hypersetup{
colorlinks,%
citecolor=blue,%
filecolor=blue,%
linkcolor=blue,%
urlcolor=blue%
}
%%\usepackage{url}
%\usepackage{rotating} % table rotation
%\usepackage{amssymb}
\usepackage{array}
\usepackage{threeparttable} % table note
\usepackage{multirow} % newline on the tabular of a table.
%%%%%%%%%%%%%%
%---
\usepackage{tabularx}
\usepackage{url}
\usepackage[usenames,dvipsnames]{xcolor}
\usepackage{pgf}
\usepackage{tikz}
% \usetikzlibrary{arrows,automata,positioning}
\usetikzlibrary{arrows,shapes}
%
% listing setting
\lstset{
breaklines=true,% 過長的程式行可斷行
extendedchars=false,% 中文處理不需要 extendedchars
texcl=true,% 中文註解需要有 TeX 處理過的 comment line, 所以設成 true
comment=[l]\%\%,% 以雙「百分號」做為程式中文註解的起頭標記,配合 MATLAB
basicstyle=\small,% 小號字體, 約 10 pt 大小
commentstyle=\upshape,% 預設是斜體字,會影響註解裏的英文,改用正體
%language=Octave % 會將一些 octave 指令以粗體顯示
}

\usepackage{url} % 在文稿中引用網址,可以用 \url{http://www.yzu.edu.tw} 方式

%%%% 以上為非必要套件
%%%%%%%%%%%p%%%%%%%%%%%%%%%%%%%

%%% 以下是 hyperref 套件
%%%%%%%%%%%%%%%%%%%%%%%%%%%%%%
% hyperref 會擾亂 cite.sty 對文獻號碼縮編的排版,所以依據
% http://www.ctan.org/tex-archive/macros/latex/contrib/hyperref/
% 作如下的更動,使得 hyperref 不做文獻號碼的超連結。
\makeatletter
\def\NAT@parse{\typeout{This is a fake Natbib command to fool Hyperref.}}
\makeatother

% hyperlinkable table of contents
% 章節目錄、圖表超連結
%%\ifx\myworkflow\mydvipdfmxflow
%%	\usepackage[dvipdfmx, debug, colorlinks, linkcolor=black, citecolor=black, urlcolor=black, unicode]{hyperref}
%%\else
%%	\usepackage[pdftex, debug, colorlinks, linkcolor=black, citecolor=black, urlcolor=black, unicode]{hyperref}	
%%\fi
%自定義
\usepackage{hyperref}
\hypersetup{colorlinks, citecolor=blue,filecolor=blue,linkcolor=blue,urlcolor=blue}

% if hyperref is not used (e.g., in LyX application)
% define dummy \phantomsection for those occurences
%   in yzu_frontpages.tex, yzu_backpages.tex, my_appendix.tex
%%\ifx\hypersetup\undefined
%%	\newcommand\phantomsection{}
%%\fi
% hyperref跟algorithm衝突,hyperref必須放在algorithm前面
\usepackage{algorithm}
%\usepackage{algorithmic}
\usepackage{algpseudocode}
\usepackage{enumerate}
%\usepackage{subfig}
%%%% 以上為所有套件
%%%%
%%%%


%% global page layout
%\newcommand{\mybaselinestretch}{1.5}  %行距 1.5 倍 + 20%, (約為 double space)
%\renewcommand{\bapselinestrpetch}{\mybapselinestretch}  % 論文行距預設值
%\parskip=2ex  % 段落之間的間隔為兩個 x 的高度
%\parindent = 0Pt  % 段首內縮由 CJpK 控制,所以這裡就設成不內縮

%%%%%%%%%%%%%%%%%%%%%%%%%%%%%
%  end of preamble
%%%%%%%%%%%%%%%%%%%%%%%%%%%%%

%%%%%%%%%%%%%%%%%%%%%%%%%%%%%
%  Thesis Information // cbj (在frontpage資料夾內ncku_thesis_cover 設定)
%%%%%%%%%%%%%%%%%%%%%%%%%%%%%

%%\renewcommand{\enTitle}{An Edge Enhancement Algorithm for Upscaled Images}  % 英文標題
%%\renewcommand{\zhTitle}{針對放大後影像之邊緣增強演算法}  %中文標題
%%\renewcommand{\authorZhName}{陳宏哲}  %作者中文姓名
%%\renewcommand{\authorEnName}{Hong-Jhe Chen}  %作者英文姓名
%%%\renewcommand{\authorStudentID}{Q36991097}  %作者學號
%%\renewcommand{\advisorZhName}{戴顯權}  %指導教授中文姓名
%%\renewcommand{\advisorEnName}{Shen-Chuan Tai}  %指導教授英文姓名
%%\renewcommand{\zhUniv}{國立成功大學}
%%\renewcommand{\enUniv}{National Cheng Kung University}
%%%\renewcommand{\zhCollegeName}{電機資訊學院}  %學院中文名稱
%%%\renewcommand{\enCollegeName}{College of Electrical Enginnering and Computer Science}  % 學院英文名稱
%%\renewcommand{\zhDepartmentName}{電腦與通信工程研究所}  %系所中文名稱
%%\renewcommand{\enDepartmentName}{Institute of Computer and Communication Engineering Department of Electrical Engineering}  % 系所英文名稱
%%\renewcommand{\rocYear}{一〇一}  %中華民國紀年年份
%%\renewcommand{\zhMonth}{六}  %中文月份
%%\renewcommand{\enYear}{2012}  %公元紀年
%%\renewcommand{\enMonth}{June}  %英文月份
%%\renewcommand{\oralDate}{101 年 6 月 19 日}  %口試日期

%
\begin{document}
\begin{CJK}{UTF8}{bkai}   %%% ZZZ %%%  <<< 在這裡更改預設中文字型、編碼 // 設定標楷體字型 // cbj
% 編碼:UTF8, Bg5, ...
% 中文字型名稱:TeXLive 安裝有一套明體字 bsmi, 楷書與其他字型視你的 LaTeX CJK 系統裝設情況而定

% 針對 latex + dvipdfmx 工作流程在 hyperref 套件的影響下,圖檔的辨識力退化
% 所作的權宜措施。可能是因為 TeXLive2007 hyperref 裏的
% 客製 graphicx / dvipdfmx 的設定檔不夠新
\ifx\myworkflow\mydvipdfmxflow
	\DeclareGraphicsExtensions{.pdf,.png,.jpg,.eps}
	\DeclareGraphicsRule{.pdf}{eps}{.bb}{}
	\DeclareGraphicsRule{.png}{eps}{.bb}{}
	\DeclareGraphicsRule{.jpg}{eps}{.bb}{}
\fi

% global CJK setting
\CJKindent  %%% ZZZ %%%  段首內縮兩格

% 載入中文名詞的定義:例如,Figure -->「圖」, Chapter -->「第 x 章」
%
% this file is encoded in utf-8
% v2.0 (Apr. 5, 2009)

% 下列中文名詞的定義,如果以註解方式關閉取消,
% 則會以系統原先的預設值 (英文) 替代
% 名詞 \prechaptername 預設值為 Chapter
% 名詞 \postchaptername 預設值為空字串
% 名詞 \tablename 預設值為 Table
% 名詞 \figurename 預設值為 Figure
%\renewcommand\prechaptername{第} % 出現在每一章的開頭的「第 x 章」
%\renewcommand\postchaptername{章}
%\renewcommand{\tablename}{表} % 在文章中 table caption 會以「表 x」表示
%\renewcommand{\figurename}{圖} % 在文章中 figure caption 會以「圖 x」表示

% 下列中文名詞的定義,用於論文固定的各部分之命名 (出現於目錄與該頁標題)
\newcommand{\nameInnerCover}{書名頁}
\newcommand{\nameCommitteeForm}{論文口試委員審定書}
\newcommand{\nameCopyrightForm}{授權書}
\newcommand{\nameCabstract}{中文摘要}
\newcommand{\nameEabstract}{Abstract} %大寫用於內文
\newcommand{\nameEabstractc}{Abstract} %小寫用於目錄c
\newcommand{\nameAckn}{ACKNOWLEDGEMENTS}
\newcommand{\nameAcknc}{Acknowledgements}
\newcommand{\nameToc}{CONTENTS}
\newcommand{\nameTocc}{Contents}
\newcommand{\nameLot}{LIST OF TABLES}
\newcommand{\nameLotc}{List of Tables}
\newcommand{\nameTof}{LIST OF FIGURES}
\newcommand{\nameTofc}{List of Figures}
\newcommand{\nameSlist}{LIST OF SYMBOLS}
\newcommand{\nameSlistc}{List of Symbols}
\newcommand{\nameVita}{VITA}
\newcommand{\nameVitac}{Vita}
 % 主標題名稱定義 // 摘要 (Abstract), ... 等 // cbj

% 如果不需要以中文數字一、二、三呈現章別,例如「第一章」
% 則請把下列介於 >>> 與 <<< 之間
% 的文字行關掉 (行首加上百分號), 會以「第 1 章」呈現
%% 中文數字章別 >>>
%\input{yzu_chnum.tex}
%% <<< 中文數字章別

%%% 以下是載入前頁、本文、後頁
%====================
%  Front Pages // cbj
%  1. 封面頁 2. 口委中英文簽名單 3. 誌謝 4. 中英文摘要
%  5. 論文目錄 6. 圖表目錄 7. 符號說明
%  1&2定義於nckuee.sty ; 3&4&7放置在資料夾 frontpages\目錄下)
%====================
%% // nckuee.sty 定義 // cbj

% 產生論文封面
\nckuEEtitlepage
% 產生口試委員會簽名單
%\nckuEEoralpage
% 產生口試委員簽名單(en)
%\nckuEEenoralpage

%\newpage
%\setcounter{page}{1}
%\pagenumbering{roman}

%%%%%%%%%%%%%%%%%%%%%%%%%%%%%%%
%       封面內頁
%%%%%%%%%%%%%%%%%%%%%%%%%%%%%%%
% % unmark to add inner cover
%\newpage
%\thispagestyle{empty}
%\thispagestyle{EmptyWaterMarkPage}
%\nckuEEtitlepage


%%%%%%%%%%%%%%%%%%%%%%%%%%%%%%%
%       中文摘要
%%%%%%%%%%%%%%%%%%%%%%%%%%%%%%%

% 可以利用如下自定義的command (定義在nckuee.sty)
% ======
%\begin{zhAbstract}  %中文摘要
%中文版簡介。手動換行會自動變成下一段文字區塊。

\begin{flushleft}
\mbox{{\bf 關鍵字}: 關鍵字1、關鍵字2、關鍵字3}
\end{flushleft}
 % // 可以引入front_cabstract.tex檔案或在此編輯 // cbj
%\end{zhAbstract}

% ...等
% ======

% 在此直接定義如下
%%%%%%%%%%%%%%%%
%
\newpage
% // HongJhe 頁碼起始
\setcounter{page}{1}
\pagenumbering{roman}
% create an entry in table of contents for 中文摘要
\phantomsection % for hyperref to register this
\addcontentsline{toc}{chapter}{\nameCabstract}
% aligned to the center of the page
\begin{center}
% font size (relative to 12 pt):
% \large (14pt) < \Large (18pt) < \LARGE (20pt) < \huge (24pt)< \Huge (24 pt)
% Set the line spacing to single for the names (to compress the lines)
\renewcommand{\baselinestretch}{1}   %行距 1 倍
% it needs a font size changing command to be effective
\LARGE{\zhTitle}\\  %中文題目
\vspace{0.83cm}
% \makebox is a text box with specified width;
% option s: stretch
% use \makebox to make sure
% each text field occupies the same width
%\makebox[1.5cm][c]{\large{學生:}}
\hspace{0.5in}
\renewcommand{\thefootnote}{\fnsymbol{footnote}}
\makebox[3.5cm][l]{\large{\authorZhName\footnote[1]{}}}\footnotetext[1]{{學生}} % 學生中文姓名
%\hfill
%
%\makebox[3cm][c]{\large{指導教授:}}
\makebox[3.5cm][l]{\large{\advisorZhName\footnote[2]{}}}\footnotetext[2]{{指導教授}} \\ %指導教授中文姓名
%
\vspace{0.42cm}
%
\large{\zhUniv}\large{\zhDepartmentName}\\ %校名系所名
\vspace{0.83cm}
%\vfill
\makebox[2.7cm][c]{\large{摘要}}
\end{center}
% Resume the line spacing to the desired setting
\renewcommand{\baselinestretch}{\mybaselinestretch}   %恢復原設定
%it needs a font size changing command to be effective
% restore the font size to normal
\normalsize
%%%%%%%%%%%%%
\par  % 摘要首段空格 by SianJhe
中文版簡介。手動換行會自動變成下一段文字區塊。

\begin{flushleft}
\mbox{{\bf 關鍵字}: 關鍵字1、關鍵字2、關鍵字3}
\end{flushleft}
 % // 可以引入front_eabstract.tex檔案或在此編輯 // cbj



%%%%%%%%%%%%%%%%%%%%%%%%%%%%%%%
%       英文摘要
%%%%%%%%%%%%%%%%%%%%%%%%%%%%%%%
%
%[method 1]

% 可以利用如下自定義的command (定義在nckuee.sty)
% ======
%\begin{enAbstract}  %英文摘要
%Add your abstract here.

\begin{flushleft}
\mbox{{\bf Keywords}: Keyword1, Keyword2, Keyword3}
\end{flushleft} % // 可以引入front_eabstract.tex檔案或在此編輯 // cbj
%\end{enAbstract}

%[method 2]
\newpage
% create an entry in table of contents for 英文摘要
\phantomsection % for hyperref to register this
\addcontentsline{toc}{chapter}{\nameEabstract} % // HongJhe marked

% aligned to the center of the page
\begin{center}
% font size:
% \large (14pt) < \Large (18pt) < \LARGE (20pt) < \huge (24pt)< \Huge (24 pt)
% Set the line spacing to single for the names (to compress the lines)
\renewcommand{\baselinestretch}{1}   %行距 1 倍
%\large % it needs a font size changing command to be effective
\LARGE{\enTitle}\\  %英文題目
\vspace{0.83cm}
% \makebox is a text box with specified width;
% option s: stretch
% use \makebox to make sure
% each text field occupies the same width
%\makebox[2cm][s]{\large{Student: }}
\hspace{0.45in}
\renewcommand{\thefootnote}{\fnsymbol{footnote}}
\makebox[5cm][l]{\large{\authorEnName\footnote[1]{}}}\footnotetext[1]{{Student}} % 學生英文姓名
%\hfill
%
%\makebox[2cm][s]{\large{Advisor: }}
\makebox[5cm][l]{\large{\advisorEnName\footnote[2]{}}}\footnotetext[2]{{Advisor}} \\ %教授英文姓名
%
\vspace{0.42cm}
\large{\enDepartmentName}\\ %英文系所全名
%
\large{\enUniv}\\  %英文校名
\vspace{0.83cm}
%\vfill
%
\large{\nameEabstractc}\\
%\vspace{0.5cm}
\end{center}

% Resume the line spacing the desired setting
\renewcommand{\baselinestretch}{\mybaselinestretch}   %恢復原設定
%\large %it needs a font size changing command to be effective
% restore the font size to normal
\normalsize
%%%%%%%%%%%%%
Add your abstract here.

\begin{flushleft}
\mbox{{\bf Keywords}: Keyword1, Keyword2, Keyword3}
\end{flushleft} % // 可以引入front_eabstract.tex檔案或在此編輯 // cbj


%%%%%%%%%%%%%%%%%%%%%%%%%%%%%%%
%       誌謝
%%%%%%%%%%%%%%%%%%%%%%%%%%%%%%%
%
% Acknowledgment
\newpage
\phantomsection % for hyperref to register this
%\addcontentsline{toc}{chapter}{\nameAcknc}

\begin{zhAckn}  %誌謝
Add your acknowledgements here.

\begin{flushright}
\mbox{Syu-Min Cyu}
\end{flushright} % // 可以引入front_ackn.tex檔案或在此編輯 // cbj
\end{zhAckn}

%\chapter*{\nameAckn} %\makebox{} is fragile; need protect
%Add your acknowledgements here.

\begin{flushright}
\mbox{Syu-Min Cyu}
\end{flushright} % // 可以引入my_ackn.tex檔案或在此編輯 // cbj
%%testjsjtoejiojsoijtoijos

%%%%%%%%%%%%%%%%%%%%%%%%%%%%%%%
%       目錄
%%%%%%%%%%%%%%%%%%%%%%%%%%%%%%%
%
% Table of contents
\newpage
\renewcommand{\contentsname}{\nameToc}
%\makebox{} is fragile; need protect
\phantomsection % for hyperref to register this
\addcontentsline{toc}{chapter}{\nameTocc}
\tableofcontents

%%%%%%%%%%%%%%%%%%%%%%%%%%%%%%%
%       表目錄
%%%%%%%%%%%%%%%%%%%%%%%%%%%%%%%
%
% List of Tables
\newpage
\renewcommand{\listtablename}{\nameLot}
%\makebox{} is fragile; need protect
\phantomsection % for hyperref to register this
\addcontentsline{toc}{chapter}{\nameLotc}
\listoftables

%%%%%%%%%%%%%%%%%%%%%%%%%%%%%%%
%       圖目錄
%%%%%%%%%%%%%%%%%%%%%%%%%%%%%%%
%
% List of Figures
\newpage
\renewcommand{\listfigurename}{\nameTof}
%\makebox{} is fragile; need protect
\phantomsection % for hyperref to register this
\addcontentsline{toc}{chapter}{\nameTofc}
\listoffigures
%%%%%%%%%%%%%%%%%%%%%%%%%%%%%%%
%       符號說明
%%%%%%%%%%%%%%%%%%%%%%%%%%%%%%%
%
% Symbol list
% define new environment, based on standard description environment
% adapted from p.60~64, <<The LaTeX Companion>>, 1994, ISBN 0-201-54199-8

%\newcommand{\SymEntryLabel}[1]%
%  {\makebox[3cm][l]{#1}}
%%
%\newenvironment{SymEntry}
%   {\begin{list}{}%
%       {\renewcommand{\makelabel}{\SymEntryLabel}%
%        \setlength{\labelwidth}{3cm}%
%        \setlength{\leftmargin}{\labelwidth}%
%        }%
%   }%
%   {\end{list}}
%%%
%\newpage
%\chapter*{\nameSlist} %\makebox{} is fragile; need protect
%\phantomsection % for hyperref to register this
%\addcontentsline{toc}{chapter}{\nameSlistc}
%\input{frontpages/front_symbols.tex}

\newpage
\setcounter{page}{1}
\pagenumbering{arabic}

%====================
%  Main Pages // cbj
%====================
% 本文
%\usepackage{geometry}  % for easy margin settings
%% margins setting // NCKU內頁設定 // cbj
%\geometry{verbose,a4paper,tmargin=2.3cm,bmargin=3.5cm,lmargin=2.5cm,rmargin=3cm}


\chapter{Introduction} \label{ch:1-introduction}
	\hspace{24pt}

Add your introduction here.

\chapter{Related Works} \label{ch:2-background}
	\hspace{24pt}

Add your related works here.

\chapter{Proposed Scheme} \label{ch:3-proposed}
	\hspace{24pt}

Add your algorithms/architecture design here.
	
\chapter{Performance Evaluation} \label{ch:4-simulation}
	\hspace{24pt}

Add your performance evalution here.
	
\chapter{Conclusions} \label{ch:5-conclusion}
	\hspace{24pt}

Add your conclusions here.


%====================
%  Back Pages // cbj
%  1. 參考文獻 2. 附錄 3. 作者簡歷
%====================
%%% 參考文獻
\newpage
\phantomsection % for hyperref to register this
\addcontentsline{toc}{chapter}{References}
\renewcommand{\bibname}{\protect\makebox[5cm][s]{REFERENCES}}
%  \makebox{} is fragile; need protect
\bibliographystyle{unsrt}
%\bibliographystyle{IEEEtranS}  % 使用 IEEE Trans 期刊格式
%  \bibliography{my_bib}

%% References with bibTeX database:
%\bibliographystyle{model1-num-names}
\bibliography{back/ncku_thesis_bib}
\baselineskip=18pt
%%\begin{thebibliography}{29}
%%\expandafter\ifx\csname natexlab\endcsname\relax\def\natexlab#1{#1}\fi
%%\providecommand{\bibinfo}[2]{#2}
%%\ifx\xfnm\relax \def\xfnm[#1]{\unskip,\space#1}\fi
%%%Type = Article
%%\bibitem[{Wiegand et~al.(2003)Wiegand, Sullivan, Bjontegaard, and Luthra}]{Wiegand2003}
%%\bibinfo{author}{T.~Wiegand}, \bibinfo{author}{G.~J. Sullivan}, \bibinfo{author}{G.~Bjontegaard}, \bibinfo{author}{A.~Luthra},
%%\newblock \bibinfo{title}{{Overview of the {H.264/AVC} video coding standard}},
%%\newblock \bibinfo{journal}{{\it IEEE Transactions on Circuits and Systems for Video Technology}},
%%\bibinfo{volume}{{\it 13}}(\bibinfo{issue}{7}), \bibinfo{year}{2003}, \bibinfo{pages}{560--576}.
%%%Type = Article
%%\bibitem[{Ostermann et~al.(2004)Ostermann, Bormans, List, Marpe, Narroschke,
%%  Pereira, Stockhammer, and Wedi}]{Ostermann2004}
%%\bibinfo{author}{J.~Ostermann}, \bibinfo{author}{J.~Bormans},
%%  \bibinfo{author}{P.~List}, \bibinfo{author}{D.~Marpe},
%%  \bibinfo{author}{M.~Narroschke}, \bibinfo{author}{F.~Pereira},
%%  \bibinfo{author}{T.~Stockhammer}, \bibinfo{author}{T.~Wedi},
%%\newblock \bibinfo{title}{{Video coding with {H.264/AVC}: tools, performance,
%%  and complexity}},
%%\newblock \bibinfo{journal}{{\it IEEE Circuits and Systems Magazine}},
%%\bibinfo{volume}{{\it 4}}(\bibinfo{issue}{1}), \bibinfo{year}{2004}, \bibinfo{pages}{7--28}.
%%%Type = Article
%%\bibitem[{Marpe et~al.(2006)Marpe, Wiegand, and Sullivan}]{Marpe2006}
%%\bibinfo{author}{D.~Marpe}, \bibinfo{author}{T.~Wiegand},
%%  \bibinfo{author}{G.~J. Sullivan},
%%\newblock \bibinfo{title}{{The {H.264/MPEG4} advanced video coding standard and its applications}},
%%\newblock \bibinfo{journal}{{\it IEEE Communications Magazine}},
%%\bibinfo{volume}{{\it 44}}(\bibinfo{issue}{8}), \bibinfo{year}{2006}, \bibinfo{pages}{134--143}.
%%%Type = Article
%%\bibitem[{Huang et~al.(2005)Huang, Hsieh, Chen, and Chen}]{Huang2005}
%%\bibinfo{author}{Y.-W. Huang}, \bibinfo{author}{B.-Y. Hsieh},
%%\bibinfo{author}{T.-C. Chen}, \bibinfo{author}{L.-G. Chen},
%%\newblock \bibinfo{title}{{Analysis, fast algorithm, and VLSI architecture
%%  design for {H.264/AVC} intra frame coder}},
%%\newblock \bibinfo{journal}{{\it IEEE Transactions on Circuits and Systems for Video Technology}},
%%\bibinfo{volume}{{\it 15}}(\bibinfo{issue}{3}), \bibinfo{year}{2005}, \bibinfo{pages}{378--401}.
%%%Type = Inproceedings
%%\bibitem[{Yu(2004)}]{Yu2004}
%%\bibinfo{author}{A.~C. Yu},
%%\newblock \bibinfo{title}{{Efficient block-size selection algorithm for
%%  inter-frame coding in {H.264/MPEG-4 AVC}}},
%%\newblock \bibinfo{booktitle}{{\it Proc. IEEE Conf. on
%%  Acoustics, Speech, and Signal Processing (ICASSP2004)}}, Montreal, QC, CA, \bibinfo{year}{2004}, \bibinfo{pages}{iii--169--72}.
%%%Type = Article
%%\bibitem[{Wu et~al.(2005)Wu, Pan, Lim, Wu, Li, Lin, Rahardja, and Ko}]{Wu2005}
%%\bibinfo{author}{D.~Wu}, \bibinfo{author}{F.~Pan}, \bibinfo{author}{K.~P. Lim},
%%  \bibinfo{author}{S.~Wu}, \bibinfo{author}{Z.~G. Li},
%%  \bibinfo{author}{X.~Lin}, \bibinfo{author}{S.~Rahardja},
%%  \bibinfo{author}{C.~C. Ko},
%%\newblock \bibinfo{title}{{Fast intermode decision in {H.264/AVC} video
%%  coding}},
%%\newblock \bibinfo{journal}{{\it IEEE Transactions on Circuits and Systems for Video Technology}},
%%  \bibinfo{volume}{{\it 15}}(\bibinfo{issue}{7}), \bibinfo{year}{2005}, \bibinfo{pages}{953--958}.
%%%Type = Article
%%\bibitem[{Grecos and Yang(2006)}]{Grecos2006}
%%\bibinfo{author}{C.~Grecos}, \bibinfo{author}{M.~Yang},
%%\newblock \bibinfo{title}{{Fast mode prediction for the baseline and main
%%  profiles in the {H.264} video coding standard}},
%%\newblock \bibinfo{journal}{{\it IEEE Transactions on Multimedia}},
%%\bibinfo{volume}{{\it 8}}(\bibinfo{issue}{7}), \bibinfo{year}{2006}, \bibinfo{pages}{1125--1134}.
%%%Type = Article
%%\bibitem[{Kannangara et~al.(2006)Kannangara, Richardson, Bystrom, Solera, Zhao, MacLennan, and Cooney}]{Kannangara2006}
%%\bibinfo{author}{C.~S. Kannangara}, \bibinfo{author}{I.~E.~G. Richardson},
%%  \bibinfo{author}{M.~Bystrom}, \bibinfo{author}{J.~R. Solera},
%%  \bibinfo{author}{Y.~Zhao}, \bibinfo{author}{A.~MacLennan},
%%  \bibinfo{author}{R.~Cooney},
%%\newblock \bibinfo{title}{{Low-complexity skip prediction for {H.264} through
%%  Lagrangian cost estimation}},
%%\newblock \bibinfo{journal}{{\it IEEE Transactions on Circuits and Systems for Video Technology}},
%%  \bibinfo{volume}{{\it 16}}(\bibinfo{issue}{2}), \bibinfo{year}{2006}, \bibinfo{pages}{202--208}.
%%%Type = Article
%%\bibitem[{Bharanitharan et~al.(2008)Bharanitharan, Liu, Yang, and
%%  Tsai}]{Bharanitharan2008}
%%\bibinfo{author}{K.~Bharanitharan}, \bibinfo{author}{B.-D. Liu},
%%  \bibinfo{author}{J.-F. Yang}, \bibinfo{author}{W.-C. Tsai},
%%\newblock \bibinfo{title}{{A low complexity detection of discrete cross
%%  differences for fast {H.264/AVC} intra prediction}},
%%\newblock \bibinfo{journal}{{\it IEEE Transactions on Multimedia}},
%% \bibinfo{volume}{{\it 10}}(\bibinfo{issue}{7}), \bibinfo{year}{2008}, \bibinfo{pages}{1250--1260}.
%%%Type = Article
%%\bibitem[{Lee and Lin(2009)}]{Lee2009}
%%\bibinfo{author}{Y.-M. Lee}, \bibinfo{author}{Y.~Lin},
%%\newblock \bibinfo{title}{{Zero-block mode decision algorithm for {H.264/AVC}}},
%%\newblock \bibinfo{journal}{IEEE Transactions on Image Processing},
%%\bibinfo{volume}{{\it 18}}(\bibinfo{issue}{3}), \bibinfo{year}{2009} \bibinfo{pages}{524--533}.
%%%Type = Article
%%\bibitem[{Zeng et~al.(2009)Zeng, Cai, and Ma}]{Zeng2009}
%%\bibinfo{author}{H.~Zeng}, \bibinfo{author}{C.~Cai}, \bibinfo{author}{K.-K.
%%  Ma},
%%\newblock \bibinfo{title}{{Fast mode decision for {H.264/AVC} based on
%%  macroblock motion activity}},
%%\newblock \bibinfo{journal}{{\it IEEE Transactions on Circuits and Systems for Video Technology}},
%%  \bibinfo{volume}{{\it 19}}(\bibinfo{issue}{4}), \bibinfo{year}{2009}, \bibinfo{pages}{491--499}.
%%%Type = Article
%%\bibitem[{Moon et~al.(2005)Moon, Kim, and Kim}]{Moon2005}
%%\bibinfo{author}{Y.~H. Moon}, \bibinfo{author}{G.~Y. Kim},
%%  \bibinfo{author}{J.~H. Kim},
%%\newblock \bibinfo{title}{{An improved early detection algorithm for all-zero
%%  blocks in {H.264} video encoding}},
%%\newblock \bibinfo{journal}{{\it IEEE Transactions on Circuits and Systems for Video Technology}},
%%  \bibinfo{volume}{{\it 15}}(\bibinfo{issue}{8}), \bibinfo{year}{2005}, \bibinfo{pages}{1053--1057}.
%%%Type = Article
%%\bibitem[{Wang et~al.(2006)Wang, Kwong, and Kok}]{Wang2006}
%%\bibinfo{author}{H.~Wang}, \bibinfo{author}{S.~Kwong}, \bibinfo{author}{C.-W. Kok},
%%\newblock \bibinfo{title}{{Efficient prediction algorithm of integer {DCT}
%%  coefficients for {H.264/AVC}optimization}},
%%\newblock \bibinfo{journal}{{\it IEEE Transactions on Circuits and Systems for Video Technology}},
%%  \bibinfo{volume}{{\it 16}}(\bibinfo{issue}{4}), \bibinfo{year}{2006}, \bibinfo{pages}{547--552}.
%%%Type = Article
%%\bibitem[{Wang and Kwong(2007)}]{Wang2007}
%%\bibinfo{author}{H.~Wang}, \bibinfo{author}{S.~Kwong},
%%\newblock \bibinfo{title}{{Hybrid model to detect zero quantized {DCT}
%%  coefficients in {H.264}}},
%%\newblock \bibinfo{journal}{{\it IEEE Transactions on Multimedia}},
%% \bibinfo{volume}{{\it 9}}(\bibinfo{issue}{4}), \bibinfo{year}{2007}, \bibinfo{pages}{728--735}.
%%%Type = Article
%%\bibitem[{Xie et~al.(2007)Xie, Liu, Liu, and Yang}]{Xie2007}
%%\bibinfo{author}{Z.~Xie}, \bibinfo{author}{Y.~Liu}, \bibinfo{author}{J.~Liu},
%%  \bibinfo{author}{T.~Yang},
%%\newblock \bibinfo{title}{{A general method for detecting all-zero blocks prior
%%  to {DCT} and quantization}},
%%\newblock \bibinfo{journal}{{\it IEEE Transactions on Circuits and Systems for Video Technology}},
%%  \bibinfo{volume}{{\it 17}}(\bibinfo{issue}{2}), \bibinfo{year}{2007}, \bibinfo{pages}{237--241}.
%%%Type = Article
%%\bibitem[{Sousa(2000)}]{Sousa2000}
%%\bibinfo{author}{L.~A. Sousa},
%%\newblock \bibinfo{title}{{General method for eliminating redundant computations in video coding}},
%%\newblock \bibinfo{journal}{{\it Electronics Letters}},
%%\bibinfo{volume}{36}(\bibinfo{issue}{2}), \bibinfo{year}{2000}, \bibinfo{pages}{306--307}.
%%%Type = Article
%%\bibitem[{Pao and Sun(1999)}]{Pao1999}
%%\bibinfo{author}{I.-M. Pao}, \bibinfo{author}{M.-T. Sun},
%%\newblock \bibinfo{title}{{Modeling {DCT} coefficients for fast video
%%  encoding}},
%%\newblock \bibinfo{journal}{{\it IEEE Transactions on Circuits and Systems for Video Technology}},
%%  \bibinfo{volume}{{\it 9}}(\bibinfo{issue}{4}),\bibinfo{year}{1999},\bibinfo{pages}{608--616}.
%%%Type = Manual
%%\bibitem[{Sta(2007)}]{StandardAVC}
%%\bibinfo{title}{{{ITU}-T Recommendation H.264 : Advanced video coding for
%%  generic audiovisual services}}, \bibinfo{organization}{International
%%  Telecommunications Union}, \bibinfo{year}{2007}.
%%%Type = Book
%%\bibitem[{Richardson(2003)}]{Richardson2003}
%%\bibinfo{author}{I.~Richardson}, \bibinfo{title}{{\it H.264 and MPEG-4 video
%%  compression: video coding for next-generation multimedia}},
%%  (\bibinfo{publisher}{Wiley}, \bibinfo{year}{2003}).
%%%Type = Article
%%\bibitem[{Lam and Goodman(2000)}]{Lam2000}
%%\bibinfo{author}{E.~Y. Lam}, \bibinfo{author}{J.~W. Goodman},
%%\newblock \bibinfo{title}{{A mathematical analysis of the DCT coefficient
%%  distributions for images}},
%%\newblock \bibinfo{journal}{{\it IEEE Transactions on Image Processing}},
%% \bibinfo{volume}{{\it 9}}(\bibinfo{issue}{10}), \bibinfo{year}{2000}, \bibinfo{pages}{1661--1666}.
%%%Type = Article
%%\bibitem[{Reininger(1983)}]{Reininger1983}
%%\bibinfo{author}{R.~Reininger} \& \bibinfo{author}{J.~Gibson},
%%\newblock \bibinfo{title}{{Distribution of the two dimensional dct coefficients
%%  for images}},
%%\newblock \bibinfo{journal}{{\it IEEE Transactions on Communications}},
%% \bibinfo{volume}{{\it 31}}(\bibinfo{issue}{6}), \bibinfo{year}{1983}, \bibinfo{pages}{835--839}.
%%%Type = Inproceedings
%%\bibitem[{Altunbasak and Kamaci(2004)}]{Altunbasak2004}
%%\bibinfo{author}{Y.~Altunbasak}, \bibinfo{author}{N.~Kamaci},
%%\newblock \bibinfo{title}{{An analysis of the DCT coefficient distribution with
%%  the H.264 video coder}},
%%\newblock \bibinfo{booktitle}{{\it Proc. IEEE Conf. on Acoustics, Speech, and Signal Processing (ICASSP2004)}}, Montreal, QC, CA, \bibinfo{year}{2004},
%%\bibinfo{pages}{iii --177--80}.
%%%Type = Book
%%\bibitem[{Eude et~al.(1994{\natexlab{a}})Eude, Cherifi, and Grisel}]{Eude1994a}
%%\bibinfo{author}{T.~Eude}, \bibinfo{author}{H.~Cherifi},
%%  \bibinfo{author}{R.~Grisel}, \bibinfo{title}{{Statistical distribution of DCT
%%  coefficients and their application to an adaptive compression algorithm}},
%%  \newblock \bibinfo{booktitle}{{\it IEEE Region 10 Conf. (TENCON1994)}}, \bibinfo{year}{1994}, Singapore, Singapore, \bibinfo{pages}{427--430}.
%%%Type = Inproceedings
%%\bibitem[{Eude et~al.(1994{\natexlab{b}})Eude, Grisel, Cherifi, and
%%  Debrie}]{Eude1994b}
%%\bibinfo{author}{T.~Eude}, \bibinfo{author}{R.~Grisel},
%%  \bibinfo{author}{H.~Cherifi}, \bibinfo{author}{R.~Debrie},
%%\newblock \bibinfo{title}{{On the distribution of the DCT coefficients}},
%%\newblock \bibinfo{booktitle}{{\it Proc. IEEE Conf. on Acoustics, Speech, and Signal Processing (ICASSP1994)}},
%%  \bibinfo{year}{1994}, Adelaide, SA, AU, \bibinfo{pages}{v--365--368}.
%%%Type = Article
%%\bibitem[{Xie and Chia(2008)}]{Xie2008}
%%\bibinfo{author}{J.~Xie}, \bibinfo{author}{L.-T. Chia},
%%\newblock \bibinfo{title}{{\it Study on the distribution of DCT residues and its
%%  application to R-D analysis of video coding}},
%%\newblock \bibinfo{journal}{Journal of Visual Communication and Image Representation},
%% \bibinfo{volume}{19}(\bibinfo{issue}{7}), \bibinfo{year}{2008}, \bibinfo{pages}{411--425}.
%%%Type = Book
%%\bibitem[{Jain(1989)}]{Jain1989}
%%\bibinfo{author}{A.~Jain}, \bibinfo{title}{{\it Fundamentals of digital image
%%  processing}} (\bibinfo{publisher}{Inc. Upper Saddle River, NJ:Prentice-Hall}, \bibinfo{year}{1989}).
%%%Type = Article
%%\bibitem[{Hang and Chen(1997)}]{Hang1997}
%%\bibinfo{author}{H.-M. Hang}, \bibinfo{author}{J.-J. Chen},
%%\newblock \bibinfo{title}{{Source model for transform video coder and its
%%  application - Part I: Fundamental theory}},
%%\newblock \bibinfo{journal}{{\it IEEE Transactions on Circuits and Systems for Video Technology}},
%% \bibinfo{volume}{{\it 7}}(\bibinfo{issue}{2}), \bibinfo{year}{1997}, \bibinfo{pages}{287--298}.
%%%Type = Article
%%\bibitem[{Ma(2005)}]{Ma2005}
%%\bibinfo{author}{S.~Ma},
%%\newblock \bibinfo{title}{{Rate-distortion analysis for H.264/AVC video coding
%%  and its application to rate control}},
%%\newblock \bibinfo{journal}{{\it IEEE Transactions on Circuits and Systems for Video Technology}},
%% \bibinfo{volume}{{\it 15}}(\bibinfo{issue}{12}), \bibinfo{year}{2005}, \bibinfo{pages}{1533--1544}.
%%%Type = Manual
%%\bibitem[{JMS(2008)}]{JMSoftware}
%%\bibinfo{title}{{{H.264/AVC} reference software, Joint Model (JM), [{O}nline]. Available:
%%  http://iphome.hhi.de/suehring/tml/}}.

%%\end{thebibliography}


%%% 附錄
%%
% this file is encoded in utf-8
% v2.0 (Apr. 5, 2009)
%%% 每一個附錄 (附錄甲、附錄乙、...) 都要複製此段附錄編排碼做為起頭
%%% 附錄編排碼 begin >>>
\newpage
\chapter*{Appendix A: MATLAB / Octave } % 修改附錄編號與你的附錄名
\phantomsection % for hyperref to register this
\addcontentsline{toc}{chapter}{Appendix A: MATLAB / Octave} %建議此內容應與上行相同
%\setcounter{chapter}{0}  %如果用的是 TeXLive2007 則打開此行以避免錯誤
\setcounter{equation}{0} 
\setcounter{figure}{0} 
\setcounter{footnote}{0} 
\setcounter{section}{0} 
\setcounter{subsection}{0}
\setcounter{subsubsection}{0}
\setcounter{table}{0} 
\renewcommand{\thechapter}{A} % 如果是附錄乙,則內容應為{乙}
%%% <<< 附錄編排碼 end

% 附錄內容開始
% 納入程式源碼
%\lstinputlisting{example_prog_list.m}


\begin{equation}\sum_{k=1}^{n} k = \frac{n(n+1)}{2}\end{equation}

%%% 如果有附錄乙、丙、...,則在此繼續加上「附錄編排」碼
% 每一個附錄會自動以新頁開始
%%% 自傳
%\newpage
\chapter*{\nameVita} % \makebox{} is fragile; need protect
\phantomsection % for hyperref to register this
\addcontentsline{toc}{chapter}{\nameVitac}

\noindent
{\it Bo-Jhih Chen} received his B.S. and M.S. degree in I-Shou University, Kaohsiung, Taiwan (R.O.C.), in 2004 and 2006, respectively. He got his Ph.D. degree of the Institute of Computer and Communication Engineering at Department of Electrical Engineering in National Cheng Kung University, Tainan, Taiwan (R.O.C.), in 2012.
His research interests are in the areas of video coding standards, digital signal processing, and multimedia system.
\bigskip \\
\noindent {\bf Education Background}\\
\noindent Sept. 2006 -- June 2012:\\
	Ph.D., Institute of Computer and Communication Engineering, Department of Electrical Engineering, National Cheng Kung University.\\
\noindent Sept. 2004 -- June 2006:\\
	M.S., Department of Computer Science and Information Engineering, I-Shou University.\\
\noindent Sept. 2000 -- June 2004:\\
	B.S., Department of Computer Science and Information Engineering, I-Shou University.

\clearpage % to make sure all CJK characters are processed
\end{CJK}  %%% ZZZ %%%
\end{document}

